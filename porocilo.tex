\documentclass[a4paper,12pt]{article}
\usepackage[slovene]{babel}
\usepackage[utf8]{inputenc}
\usepackage[T1]{fontenc}
\usepackage{lmodern}
\usepackage{amsmath,amsfonts}
\usepackage{enumitem}
\usepackage{mathabx}
\usepackage{fancyhdr}
\usepackage{url}
\usepackage{graphicx}
\usepackage{xcolor}
\usepackage{amsthm}
\usepackage{amssymb}

\usepackage{booktabs}   % lepše črte v tabelah
\usepackage{array}      % boljši stolpci

\theoremstyle{definition}
\newtheorem{definicija}{Definicija}

\title{Metrične dimenzije usmerjenih grafov}
\author{Jasmina Mazej in Lana Stojčić}

\begin{document}
\maketitle

\section*{Definicije}

\noindent Naj bo $G = (V, E)$ usmerjen graf, kjer je $V$ množica vseh vozlišč in $E$ množica vseh usmerjenih povezav v grafu.
Privzeli bomo, da je v vseh definicijah graf usmerjen.

\begin{definicija}
    Če obstaja direktna povezava iz vozlišča $u$ v $v$, potem je to najkrajša povezava med vozliščima.
    Označimo jo z $d(u, v)$. Če taka povezava obstaja, pravimo da je $v$ dosegljiv iz $u$.
    Če taka pot ne obstaja, označimo $d(u, v) = \infty$ in $v$ ni dosegljiv iz $u$.
\end{definicija}

\begin{definicija}
    Naj bodo $s, x, y \in V$. Pravimo, da vozlišče $s$ razreši par vozlišč $x, y\in V$, če sta $x$ in $y$ obe dosegljiv
    iz $s$, vendar pri različnih razdaljah. Torej velja:
    \[d(s, x) \neq d(s, y).\]
    Množica vozlišč $S$ razreši usmerjen graf $G$, če za vsak par $x, y\in V$ obstaja vsaj eno vozlišče $s\in S$,
    ki razreši $x$ in $y$. Najmanjša možna velikost množice $S$ se imenuje metrična dimenzija grafa G.
\end{definicija}

\begin{definicija}
    Usmerjene cirkulantne grafe označimo kot $C(n, d)$, kjer je $n$ število vozlišč v grafu, $d$ pa seznam generatorjev.
    Označimo jih lahko tudi kot $C_n(1,\dots , d)$. Njihova vozlišča so postavljena v krog, povezave pa si sledijo v periodičnem
    vzorcu. Vozlišča v tem grafu bomo označevali kot ${v_0, v_1, \dots, v_{n-1}}$. Vsako vozlišče je povezano z določenim drugim po fiksnem
    vzorcu, ki ga določijo generatorji. Torej množico povezav zapišemo kot: 
    \[E = \{(v_i, v_{i + k \text{  mod n}}); \text{  }1 \leq k \leq d\}\text{   za } i = \{0, 1, \dots, n-1\}.\]
\end{definicija}

%-------------------------------------------------
\section*{Uvod in opis problema}

\subsection*{Naloga 1}

\noindent Metrična dimenzija grafa je pomembna invariantna količina v teoriji grafov, ki meri najmanjšo množico vozlišč, s pomočjo katerih lahko z razdaljami enolično ločimo vsa vozlišča grafa. 

\noindent V projektu sva obravnavali \textbf{usmerjene cirkulantne grafe}
\[
    C_n(1,2,\dots,k),
\]
ter eksperimentalno preverili znane teoretične ocene in domneve o njihovi metrični dimenziji. \\


\noindent Znano je, da velja:
\[
    \dim(C_n(1,2,\dots,k)) = k \quad \text{za } n > 2(k-1)^2.
\]

\noindent Najin cilj je preveriti, ali lahko obstaja manjša meja $f(k) < 2(k-1)^2$.

\noindent Najprej sva eksperimentalno preverili vrednosti $k \in \{3, 4, 5, 6\}$ in $n \in [4, 77] $ na že znani trditvi ter zbrali rezultate v tabeli.
Da sva lahko to preverili, sva definirali funkcijo:

\begin{verbatim}
    def preveri_dimenzijo(n, k):
        #funkcija ne deluje za k > n:
        if k >= n:
                return {"n": n, 
                    "k": k, 
                    "napaka": "Za cirkulantni graf mora veljati k < n."}
    
        # V primeru, da je n manjsi od funkcija f(k)
        if n <= 2*(k - 1)**2:
            return {"n": n, 
                    "k": k, 
                    "napaka": "Pogoj n > 2(k - 1)^2 ni izpolnjen."}

        # Ustvari cirkulantni graf
        Cn_k = clockwise_circulant_graph(n, k)

        # Izračuna razresljivo mn in pove metricno dimenzijo
        razresljiva_mnozica, dim = metricna_dimenzija_usmerjenega_grafa(Cn_k)

        # Preveri enakost
        rezultat = (dim == k)

        return {"n": n, 
                "k": k, 
                "dimCn_k": dim, 
                "trditev_velja": rezultat}
\end{verbatim}

\noindent Funkcija \texttt{preveri\_dimenzijo(n, k)} vrne metrčno dimenzijo grafa $C_n(1,2,\dots,k)$ in preveri, ali velja $\dim(C_n(1,2,\dots,k)) = k$ za podana $n$ in $k$.
Pri oblikovanju te funkcije sva si pomagali s kodo Maše Popovič iz prejšnjega projekta.
Rezultati so potrdili znano trditev za vse preizkušene vrednosti. \\


\noindent Nato sva za $f(k)$ izbrali funkcije $f(k) = 2(k-1)^2 - a$ za poljuben $a \in \{1, 2, 3, 4, 5\}$ in preverili, ali še vedno velja 
\[
    \dim(C_n(1,2,\dots,k)) = k \quad \text{za } n > f(k).
\]

\noindent Ponovno sva definirali funkcijo, kjer sva dodali parameter $a$:

\begin{verbatim}
    def preveri_dimenzijo2(n, k, a):
        #funkcija ne deluje za k > n:
        if k >= n:
            return {"n": n, 
                    "k": k, 
                    "napaka": "Za cirkulantni graf mora veljati k < n."}
    
        # V primeru, da je n manjsi od funkcija f(k)
        if n <= 2*(k - 1)**2 - a:
            return {"n": n, 
                    "k": k, 
                    "napaka": f"Pogoj n > 2(k-1)^2 - {a} ni izpolnjen."}

        # Ustvari cirkulantni graf
        Cn_k = clockwise_circulant_graph(n, k)

        # Izračuna razresljivo mn in pove metricno dimenzijo
        razresljiva_mnozica, dim = metricna_dimenzija_usmerjenega_grafa(Cn_k)

        # Preveri enakost
        rezultat = (dim == k)

        return {"n": n, 
                "k": k, 
                "dimCn_k": dim, 
                "trditev_velja": rezultat}
\end{verbatim}

\noindent Kot prej sva vzeli $n \in [4, 77]$ in $k \in \{3, 4, 5, 6\}$. Rezultati so pokazali sledeče:
\begin{itemize}
    \item za $a = 1$: enakost velja za vse $n \in [4, 77]$ in $k \in \{3, 4, 5, 6\}$,
    \item za $a = 2$: enakost velja za vse $n \in [4, 77]$ in $k \in \{3, 4, 5, 6\}$,
    \item za $a = 3$: enakost velja za vse $n \in [4, 77]$ in $k \in \{3, 4, 5, 6\}$,
    \item za $a = 4$: enakost ne velja za $n = 5$ in $k = 3$, $\dim(C_5(1, 2, 3)) = 2$ kar ni enako 3,
    \item za $a = 5$: enakost ne velja za $n = 5$ in $k = 3$, $\dim(C_5(1, 2, 3)) = 2$ kar ni enako 3.
\end{itemize}
Rezultati so pokazali, da je najmanjši $f(k)$ za katerega velja $n > f(k)$ in $\dim(C_n(1,2,\dots,k)) = k$ enak
\[f(k) = 2(k-1)^2 - 3.\]
Torej sva našli manjšo funkcijo $f(k)$ od prej znane meje $2(k-1)^2$. 
To lahko trdive le za $n \in [4, 77]$ in $k \in \{3, 4, 5, 6\}$, saj sva le na teh vrednostih preverili enakost dimenzije in $k$.\\

Kot zanimivost sva si ogledali še, kaj se zgodi če za $f(k)$ uzameve linearno funkcijo. Izbrali sva se $f(k) =  2(k - 1)$. 
Že pri $k = 3$ lahko opazimo, da je pri $n = 5$ dimenizija enaka 2, kar ni enako $k$. Vidimo tudi, da pri večjih $k$ lahko najdemo tak
$n$, da dimenzija ne bo enaka $k$. Torej linearna funkcija ne bo ustrezna izbira za $f(k)$, če želimo imeti enakost $\dim(C_n(1,2,\dots,k)) = k$.\noindent Nadalje sva eksperimentalno preverili še za $f(k) = 2(k-1)^2 - a$ , kjer je $a \in \{2, 3, 4, 5\}$. Rezultati so pokazali, da za $a \in \{2, 3\}$ dobimo pravilne vrednosti. \\

\begin{table}[ht]
    \centering
    \caption{Dimenzije cirkulantnih grafov (4. list)}
    \label{tab:cirkulantni_4}
    \begin{tabular}{@{}ccccccccc@{}}
       \toprule
        n & 3 & 4 & 5 & 6 \\
        \midrule
        4 &  &  &  &  \\
        5 &  &  &  &  \\
        6 & 3 &  &  &  \\
        7 & 3 &  &  &  \\
        8 & 3 &  &  &  \\
        14 & 3 &  &  &  \\
        15 & 3 &  &  &  \\
        16 & 3 & 4 &  &  \\
        17 & 3 & 4 &  &  \\
        21 & 3 & 4 &  &  \\
        28 & 3 & 4 &  &  \\
        29 & 3 & 4 &  &  \\
        30 & 3 & 4 & 5 &  \\
        31 & 3 & 4 & 5 &  \\
        32 & 3 & 4 & 5 &  \\
        45 & 3 & 4 & 5 &  \\
        46 & 3 & 4 & 5 &  \\
        47 & 3 & 4 & 5 &  \\
        48 & 3 & 4 & 5 & 6 \\
        49 & 3 & 4 & 5 & 6 \\
        50 & 3 & 4 & 5 & 6 \\
        \bottomrule
    \end{tabular}   
\end{table}

\noindent Pri $f(k) = 2(k-1)^2 - 4$ in $f(k) = 2(k-1)^2 - 5$ so se pojavila odstopanja, vendar le za $k=3$.\\

\noindent Za $f(k)$ sva nato vzeli še $f(k) = 2(k-1)$. Tukaj so rezultati odstopali pri vseh $k$, ki sva jih preizkušali.\\

\noindent \textbf{Sklep:}
Spodnja meja za vrednosti $n$ in $k$, ki sva jih preizkušali je $2(k-1)^2 - 3$.
    

\subsection*{Naloga 2}

Vemo, da veljajo sledeče enakosti:
\begin{itemize}
    \item $dim(C_n(1, 2, \ldots, n-1)) \geq n-1 \quad n \geq 3$
    \item $dim(C_n(1, 2, \ldots, n-2)) = \lfloor \frac{n}{2} \rfloor \quad n \geq 4$
    \item $dim(C_n(1, 2, \ldots, n-3)) = \lfloor \frac{n}{2} \rfloor \quad n \geq 5$
\end{itemize}
Želimo videti ali za $n \geq 7$ velja $dim(C_n(1, \dots, n - 4)) = \lceil \frac{2n}{5} \rceil$.\\

Enakost sva preverili tako, da sva definirali dve funkciji. Prvo, ki računa metrično dimenzijo grafa:
\begin{verbatim}
    def dimenzija_odvisna_od_n(n, m):
        if m >= n:
            return {"n": n, 
                    "m": m,
                    "dimenzija": None} 
    
        G = clockwise_circulant_graph(n, n-m)
        razresljiva_mnozica, dim = metricna_dimenzija_usmerjenega_grafa(G)
        return {"n": n, 
                "m": m,
                "dimenzija": dim}
\end{verbatim}
in drugo, ki vzame seznam vrednosti $n$:
\begin{verbatim}
    def vse_dimenzije(vrednosti_n, m):
        rezultati = [] 

        for n in vrednosti_n:
            rezultati.append(dimenzija_odvisna_od_n(n, m))
    
        return rezultati
\end{verbatim}

\noindent Na podlagi teh dveh funkcij, sva izračunali metrično dimenzijo za sve $n \in [7, 80]$ in $m = 4$.
V nalogi sva na mesto $k$ uporabljali $m = n - k$, saj je v tej nalogi $k$ odvisen od $n$.
Rezultate sva shranili v excel tabeli.

















%-------------------------------------------------
\section*{Problem 1: Iskanje funkcije $f(k)$}

Naloga:
\[
\text{za } k\ge 3 \text{ poišči } f(k)<2(k-1)^2 \text{, da velja } \dim(C_n(1,\dots,k))=k.
\]

Eksperimentalno smo s pomočjo izračunov preverili več vrednosti $k$ in $n$.

\textbf{Rezultati (iz Excel tabele):}
\begin{itemize}
\item za $k=3$: $f(3) = \underline{\hspace{1cm}}$,
\item za $k=4$: $f(4) = \underline{\hspace{1cm}}$,
\item za $k=5$: $f(5) = \underline{\hspace{1cm}}$.
\end{itemize}

(V te vrstice samo prepišeš konkretne vrednosti iz Excel tabele.)

Ugotovimo, da v vseh obravnavanih primerih velja:
\[
f(k) < 2(k-1)^2.
\]

%-------------------------------------------------
\section*{5. Znane spodnje meje}

Velja:
\[
\dim(C_n(1,\dots,n-1)) \ge n-1,\quad n\ge 3,
\]
\[
\dim(C_n(1,\dots,n-2)) \ge \left\lceil\frac{n}{2}\right\rceil,\quad n\ge 4,
\]
\[
\dim(C_n(1,\dots,n-3)) \ge \left\lceil\frac{n}{2}\right\rceil,\quad n\ge 5.
\]

Tudi te ocene se skladajo z našimi numeričnimi rezultati.

%-------------------------------------------------
\section*{6. Conjecture 2: Primer $k = n-4$}

Domneva:
\[
\dim(C_n(1,2,\dots,n-4)) = \left\lceil\frac{2n}{5}\right\rceil, \quad n\ge 7.
\]

S pomočjo izračunov za $7 \le n \le 40$ smo ugotovili:

\begin{center}
\emph{Domneva velja za vse preizkušene vrednosti $n$.}
\end{center}

(Tu lahko po želji vstaviš tudi posnetek zaslona Excel tabele.)

%-------------------------------------------------
\section*{7. Splošni primer $k \ge 5$}

Za vrednosti $k=5,6,7,8,9,10$ smo eksperimentalno izračunali vrednosti
\[
\dim(C_n(1,\dots,n-k)).
\]

Iz rezultatov (glej Excel datoteko) opazimo:
\begin{itemize}
\item metrična dimenzija narašča približno linearno z $n$,
\item večje kot je $k$, počasnejša je rast,
\item zdi se, da velja približna zveza:
\[
\dim(C_n(1,\dots,n-k)) \approx \alpha_k n.
\]
\end{itemize}

%-------------------------------------------------
\section*{8. Zaključek}

V nalogi smo uspešno:
\begin{itemize}
\item implementirali izračun metrične dimenzije,
\item eksperimentalno določili boljše meje za Problem 1,
\item potrdili veljavnost domneve za $k=n-4$,
\item ter analizirali obnašanje za splošne vrednosti $k\ge 5$.
\end{itemize}

Projekt potrjuje, da je kombinacija ILP in eksperimentalne analize učinkovito orodje pri preučevanju metrične dimenzije.

\nocite{*}

\end{document}