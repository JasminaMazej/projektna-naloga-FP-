\documentclass[a4paper,12pt]{article}
\usepackage[slovene]{babel}
\usepackage[utf8]{inputenc}
\usepackage[T1]{fontenc}
\usepackage{lmodern}
\usepackage{amsmath,amsfonts}
\usepackage{enumitem}
\usepackage{mathabx}
\usepackage{fancyhdr}
\usepackage{url}
\usepackage{graphicx}
\usepackage{xcolor}
\usepackage{amsthm}
\usepackage{amssymb}

\usepackage{booktabs}   % lepše črte v tabelah
\usepackage{array}      % boljši stolpci
\usepackage{float}     % H opcija za tabele
\usepackage{fancyvrb}  % za lepši prikaz kode
\usepackage[table]{xcolor}


\theoremstyle{definition}
\newtheorem{definicija}{Definicija}

\title{Metrične dimenzije usmerjenih grafov}
\author{Jasmina Mazej in Lana Stojčić}

\begin{document}
\maketitle

\section*{Definicije}

\noindent Naj bo $G = (V, E)$ usmerjen graf, kjer je $V$ množica vseh vozlišč in $E$ množica vseh usmerjenih povezav v grafu.
Privzeli bomo, da je v vseh definicijah graf usmerjen.

\begin{definicija}
    Če obstaja direktna povezava iz vozlišča $u$ v $v$, potem je to najkrajša povezava med vozliščima.
    Označimo jo z $d(u, v)$. Če taka povezava obstaja, pravimo da je $v$ dosegljiv iz $u$.
    Če taka pot ne obstaja, označimo $d(u, v) = \infty$ in $v$ ni dosegljiv iz $u$.
\end{definicija}

\begin{definicija}
    Naj bodo $s, x, y \in V$. Pravimo, da vozlišče $s$ razreši par vozlišč $x, y\in V$, če sta $x$ in $y$ obe dosegljiv
    iz $s$, vendar pri različnih razdaljah. Torej velja:
    \[d(s, x) \neq d(s, y).\]
    Množica vozlišč $S$ razreši usmerjen graf $G$, če za vsak par $x, y\in V$ obstaja vsaj eno vozlišče $s\in S$,
    ki razreši $x$ in $y$. Najmanjša možna velikost množice $S$ se imenuje metrična dimenzija grafa G.
\end{definicija}

\begin{definicija}
    Usmerjene cirkulantne grafe označimo kot $C(n, d)$, kjer je $n$ število vozlišč v grafu, $d$ pa seznam generatorjev.
    Označimo jih lahko tudi kot $C_n(1,\dots , d)$. Njihova vozlišča so postavljena v krog, povezave pa si sledijo v periodičnem
    vzorcu. Vozlišča v tem grafu bomo označevali kot ${v_0, v_1, \dots, v_{n-1}}$. Vsako vozlišče je povezano z določenim drugim po fiksnem
    vzorcu, ki ga določijo generatorji. Torej množico povezav zapišemo kot: 
    \[E = \{(v_i, v_{i + k \text{  mod n}}); \text{  }1 \leq k \leq d\}\text{   za } i = \{0, 1, \dots, n-1\}.\]
\end{definicija}

%-------------------------------------------------
\section*{Opis problema in postopek reševanja}

\subsection*{Naloga 1}

\noindent Metrična dimenzija grafa je pomembna invariantna količina v teoriji grafov, ki meri najmanjšo množico vozlišč, s pomočjo katerih lahko z razdaljami enolično ločimo vsa vozlišča grafa. 

\noindent V projektu sva obravnavali \textbf{usmerjene cirkulantne grafe}
\[
    C_n(1,2,\dots,k),
\]
ter eksperimentalno preverili znane teoretične ocene in domneve o njihovi metrični dimenziji. \\


\noindent Znano je, da velja:
\[
    \dim(C_n(1,2,\dots,k)) = k \quad \text{za } n > 2(k-1)^2.
\]

\noindent Najin cilj je preveriti, ali lahko obstaja manjša meja $f(k) < 2(k-1)^2$.

\noindent Najprej sva eksperimentalno preverili vrednosti $k \in \{3, 4, 5, 6\}$ in $n \in [4, 77] $ na že znani trditvi ter zbrali rezultate v tabeli.
Da sva lahko to preverili, sva definirali funkcijo:

{\scriptsize
\begin{verbatim}
    def preveri_dimenzijo(n, k):
        #funkcija ne deluje za k > n:
        if k >= n:
                return {"n": n, 
                    "k": k, 
                    "napaka": "Za cirkulantni graf mora veljati k < n."}
    
        # V primeru, da je n manjsi od funkcija f(k)
        if n <= 2*(k - 1)**2:
            return {"n": n, 
                    "k": k, 
                    "napaka": "Pogoj n > 2(k - 1)^2 ni izpolnjen."}

        # Ustvari cirkulantni graf
        Cn_k = clockwise_circulant_graph(n, k)

        # Izračuna razresljivo mn in pove metricno dimenzijo
        razresljiva_mnozica, dim = metricna_dimenzija_usmerjenega_grafa(Cn_k)

        # Preveri enakost
        rezultat = (dim == k)

        return {"n": n, 
                "k": k, 
                "dimCn_k": dim, 
                "trditev_velja": rezultat}
\end{verbatim}
}

\noindent Funkcija \texttt{preveri\_dimenzijo(n, k)} vrne metrčno dimenzijo grafa $C_n(1,2,\dots,k)$ in preveri, ali velja $\dim(C_n(1,2,\dots,k)) = k$ za podana $n$ in $k$.
Pri oblikovanju te funkcije sva si pomagali s kodo Maše Popovič iz projekta iz študijskega leta 2024/2025.
Rezultati so potrdili znano trditev za vse preizkušene vrednosti. \\


\noindent Nato sva za $f(k)$ izbrali funkcije $f(k) = 2(k-1)^2 - a$ za poljuben $a$. Preverili sva za vrednosti $a \in \{1, 2, 3, 4, 5\}$, ali še vedno velja: 
\[
    \dim(C_n(1,2,\dots,k)) = k \quad \text{za } n > f(k).
\]

\noindent Ponovno sva definirali funkcijo, kjer sva dodali parameter $a$:

{\scriptsize
\begin{verbatim}
    def preveri_dimenzijo2(n, k, a):
        #funkcija ne deluje za k > n:
        if k >= n:
            return {"n": n, 
                    "k": k, 
                    "napaka": "Za cirkulantni graf mora veljati k < n."}
    
        # V primeru, da je n manjsi od funkcija f(k)
        if n <= 2*(k - 1)**2 - a:
            return {"n": n, 
                    "k": k, 
                    "napaka": f"Pogoj n > 2(k-1)^2 - {a} ni izpolnjen."}

        # Ustvari cirkulantni graf
        Cn_k = clockwise_circulant_graph(n, k)

        # Izračuna razresljivo mn in pove metricno dimenzijo
        razresljiva_mnozica, dim = metricna_dimenzija_usmerjenega_grafa(Cn_k)

        # Preveri enakost
        rezultat = (dim == k)

        return {"n": n, 
                "k": k, 
                "dimCn_k": dim, 
                "trditev_velja": rezultat}
\end{verbatim}
}

\noindent Kot prej sva vzeli $n \in [4, 77]$ in $k \in \{3, 4, 5, 6\}$. Rezultati so pokazali sledeče:
\begin{itemize}
    \item za $a = 1$: enakost velja za vse $n \in [4, 77]$ in $k \in \{3, 4, 5, 6\}$,
    \item za $a = 2$: enakost velja za vse $n \in [4, 77]$ in $k \in \{3, 4, 5, 6\}$,
    \item za $a = 3$: enakost velja za vse $n \in [4, 77]$ in $k \in \{3, 4, 5, 6\}$,
    \item za $a = 4$: enakost ne velja za $n = 5$ in $k = 3$, $\dim(C_5(1, 2, 3)) = 2$ kar ni enako 3,
    \item za $a = 5$: enakost ne velja za $n = 5$ in $k = 3$, $\dim(C_5(1, 2, 3)) = 2$ kar ni enako 3.
\end{itemize}
Rezultati so pokazali, da je najmanjši $f(k)$ za katerega velja $n > f(k)$ in $\dim(C_n(1,2,\dots,k)) = k$ enak
\[f(k) = 2(k-1)^2 - 3.\]
Torej sva našli manjšo funkcijo $f(k)$ od prej znane meje $2(k-1)^2$. 
To lahko trdiva le za $n \in [4, 77]$ in $k \in \{3, 4, 5, 6\}$, saj sva le na teh vrednostih preverili enakost dimenzije.\\

\noindent Kot zanimivost sva si ogledali še, kaj se zgodi, če za $f(k)$ uzameva linearno funkcijo. Izbrali sva $f(k) =  2(k - 1)$. 
Že pri $k = 3$  lahko opazimo, da je pri $n = 5$ dimenizija enaka 2, kar ni enako $k$. Vidimo tudi, da pri večjih $k$ lahko najdemo tak
$n$, da dimenzija ne bo enaka $k$. Torej linearna funkcija ne bo ustrezna izbira za $f(k)$, če želimo imeti enakost $\dim(C_n(1,2,\dots,k)) = k$.\\

\noindent \textbf{Sklep:}
Spodnja meja $f(k) < 2(k-1)^2$ za vrednosti $n \in [4, 77]$ in $k \in \{3, 4, 5, 6\}$ je $f(k) = 2(k-1)^2 - 3$.\\

\noindent \textbf{Komentar:}\\
Vsi rezultati so zbrani v excel datoteki \texttt{rezultati\_dimenzija\_cirkulantnih\_grafov.xlsx}, 
kjer je vsak list poimenovan po funkciji $f(k)$. Spodnji tabeli prikazujta rezultate le za določne $n$.
Ne zajemata vseh vrednosti, saj so tabele v excelu precej velike. Njun namen je prikazati, kako opazimo da dimenzija ustreza $k$ ali ne.

\begin{table}[H]
    \centering
    \begin{minipage}{0.48\textwidth}
        \centering
        \caption{Dimenzije cirkulantnih grafov (4. list)}
        \label{tab:cirkulantni_4}
        \begin{tabular}{@{}ccccccccc@{}}
            \toprule
            n & 3 & 4 & 5 & 6 \\
            \midrule
            4 &  &  &  &  \\
            5 &  &  &  &  \\
            6 & 3 &  &  &  \\
            7 & 3 &  &  &  \\
            8 & 3 &  &  &  \\
            14 & 3 &  &  &  \\
            15 & 3 & 4 &  &  \\
            16 & 3 & 4 &  &  \\
            17 & 3 & 4 &  &  \\
            21 & 3 & 4 &  &  \\
            28 & 3 & 4 &  &  \\
            29 & 3 & 4 &  &  \\
            30 & 3 & 4 & 5 &  \\
            31 & 3 & 4 & 5 &  \\
            32 & 3 & 4 & 5 &  \\
            45 & 3 & 4 & 5 &  \\
            46 & 3 & 4 & 5 &  \\
            47 & 3 & 4 & 5 &  \\
            48 & 3 & 4 & 5 & 6 \\
            49 & 3 & 4 & 5 & 6 \\
            50 & 3 & 4 & 5 & 6 \\
            51 & 3 & 4 & 5 & 6 \\
            52 & 3 & 4 & 5 & 6 \\
            53 & 3 & 4 & 5 & 6 \\
            54 & 3 & 4 & 5 & 6 \\
            55 & 3 & 4 & 5 & 6 \\
            56 & 3 & 4 & 5 & 6 \\
            57 & 3 & 4 & 5 & 6 \\
            58 & 3 & 4 & 5 & 6 \\
            59 & 3 & 4 & 5 & 6 \\
            60 & 3 & 4 & 5 & 6 \\
            \bottomrule
        \end{tabular}   
    \end{minipage}
    \hfill
    \begin{minipage}{0.48\textwidth}
        \centering
        \caption{Dimenzije cirkulantnih grafov (5. list)}
        \label{tab:cirkulantni_5}
        \begin{tabular}{@{}ccccccccc@{}}
            \toprule
            n & 3 & 4 & 5 & 6 \\
            \midrule
            4 &  &  &  &  \\
            5 & \cellcolor{red}2 &  &  &  \\
            6 & 3 &  &  &  \\
            7 & 3 &  &  &  \\
            8 & 3 &  &  &  \\
            14 & 3 &  &  &  \\
            15 & 3 & 4 &  &  \\
            16 & 3 & 4 &  &  \\
            17 & 3 & 4 &  &  \\
            21 & 3 & 4 &  &  \\
            28 & 3 & 4 &  &  \\
            29 & 3 & 4 & 5 &  \\
            30 & 3 & 4 & 5 &  \\
            31 & 3 & 4 & 5 &  \\
            32 & 3 & 4 & 5 &  \\
            45 & 3 & 4 & 5 &  \\
            46 & 3 & 4 & 5 &  \\
            47 & 3 & 4 & 5 & 6 \\
            48 & 3 & 4 & 5 & 6 \\
            49 & 3 & 4 & 5 & 6 \\
            50 & 3 & 4 & 5 & 6 \\
            51 & 3 & 4 & 5 & 6 \\
            52 & 3 & 4 & 5 & 6 \\
            53 & 3 & 4 & 5 & 6 \\
            54 & 3 & 4 & 5 & 6 \\
            55 & 3 & 4 & 5 & 6 \\
            56 & 3 & 4 & 5 & 6 \\
            57 & 3 & 4 & 5 & 6 \\
            58 & 3 & 4 & 5 & 6 \\
            59 & 3 & 4 & 5 & 6 \\
            60 & 3 & 4 & 5 & 6 \\
            \bottomrule
        \end{tabular}
    \end{minipage}
\end{table}    

\noindent Tabela \ref{tab:cirkulantni_4} prikazuje rezultate za $f(k) = 2(k-1)^2 - 3$, medtem ko tabela \ref{tab:cirkulantni_5} prikazuje rezultate za $f(k) = 2(k-1)^2 - 4$. 
Iz tabel opazimo, da za $f(k) = 2(k-1)^2 - 3$ enakost velja za vse preizkušene vrednosti, medtem ko za $f(k) = 2(k-1)^2 - 4$ ne velja
za $n = 5$ in $k = 3$.

\newpage
\subsection*{Naloga 2}

\noindent Vemo, da veljajo sledeče enakosti:
\begin{itemize}
    \item $dim(C_n(1, 2, \ldots, n-1)) \geq n-1 \quad n \geq 3$
    \item $dim(C_n(1, 2, \ldots, n-2)) = \lfloor \frac{n}{2} \rfloor \quad n \geq 4$
    \item $dim(C_n(1, 2, \ldots, n-3)) = \lfloor \frac{n}{2} \rfloor \quad n \geq 5$
\end{itemize}
\noindent Preverili sva, ali za $n \geq 7$ velja $dim(C_n(1, \dots, n - 4)) = \lceil \frac{2n}{5} \rceil$.\\

\noindent Nato sva raziskali še, ali obstajajo podobni rezultati za $dim(C_n(1,\dots, n - k))$ za $k \geq 5$.
Da sva lažje razlikovali med to in prejšnjo nalogo, sva označili $m := k$.\\

\subsubsection*{Naloga 2.1}

\noindent Enakost $dim(C_n(1, \dots, n - 4)) = \lceil \frac{2n}{5} \rceil$ za $n \geq 7$ sva preverili tako, da sva definirali dve funkciji. 
Funkcija $\textit{dimenzija\_odvisna\_od\_n(n, m)}$ računa metrično dimenzijo grafa:

{\scriptsize
\begin{verbatim}
    def dimenzija_odvisna_od_n(n, m):
        if m >= n:
            return {"n": n, 
                    "m": m,
                    "dimenzija": None} 
    
        G = clockwise_circulant_graph(n, n-m)
        razresljiva_mnozica, dim = metricna_dimenzija_usmerjenega_grafa(G)
        return {"n": n, 
                "m": m,
                "dimenzija": dim}
\end{verbatim}
}

\noindent Funkcija $\textit{vse\_dimenzije(vrednosti\_n, m)}$ vzame seznam vrednosti $n$ in vrne seznam izračunanih metričnih dimenzij:
{\scriptsize
\begin{verbatim}
    def vse_dimenzije(vrednosti_n, m):
        rezultati = [] 

        for n in vrednosti_n:
            rezultati.append(dimenzija_odvisna_od_n(n, m))
    
        return rezultati
\end{verbatim}
}

\noindent Na podlagi teh dveh funkcij, sva izračunali metrično dimenzijo za vse $n\in[7, 80]$ in $m = 4$.
Rezultate sva shranili v excel tabelo. \\

\noindent Ustvarili sva seznam s pomočjo for zanke, ki za vsak $n$ izračuna $\lceil \frac{2n}{5} \rceil$. Dobljene vrednosti, pa sva na to primerjali
z izračunanimi metričnimi dimenzijami.
{\scriptsize
\begin{verbatim}
    #Sezam s predvideno dimenzijo
    izracun_dim = []
    for n in sez_n:
        st = ceil(2*n/5)
        izracun_dim.append(st)

    #Dimenzija za m=4
    tabela = pd.read_excel("rezultati_naloga_2.xlsx", sheet_name="dimenzija za m=4")
    vrednosti = tabela.iloc[:,1].tolist()

    rezultat = [a == b for a, b in zip(izracun_dim, vrednosti)]
\end{verbatim}
}

\noindent Rezultati so pokazali, da velja enakost med vsemi izračunanimi dimenzijami in $\lceil \frac{2n}{5} \rceil$.\\

\noindent \textbf{Sklep:}
Za vsak $n \in [7, 80]$ velja $dim(C_n(1, \dots, n - 4)) = \lceil \frac{2n}{5} \rceil$.


\subsubsection*{Naloga 2.2}
V tem delu projekta sva eksperimentalno preverili obnašanje metrične dimenzije $dim(C_n(1, \dots, n - m))$, kjer 
sva vzeli $m \in \{5, 6, 7, 8, 9, 10\}$.\\

\noindent Pri reševanju sva uporabili zgoraj omenjeni funkciji in rezultate shranili v excel datoteko. To lahko najdete na 
najnem repozitoriju pod naslovom \texttt{rezultati\_naloga\_2.xlsx}.\\

\noindent Najne ugotovitve so sledeče:
\begin{itemize}
    \item Za $m = 5$: Iz rezultatov sva opazili, da lahko metrično dimenzijo opiševa s funkcijo
    \[\dim(C_n(1,2,\dots,n-5)) =
        \begin{cases}
            \dfrac{n}{3}, & \text{če } 3 \mid n, \\[10pt]
            \left\lceil \dfrac{n}{3} \right\rceil + 1, & \text{sicer}.
        \end{cases}\]
    Definirali sva funkcijo $\textit{dimenzija\_m5(n)}$ in zračunali njene vrednosti za $n \in \{7, \ldots ,80\}$. 
    {\scriptsize
    \begin{verbatim}
        def dimenzija_m5(n):
            if n % 3 == 0:
                return n // 3
            else:
                return ceil(n / 3) + 1
    \end{verbatim}
    }
    Nato sva rezultate te funkcije    primerjali z izračunanimi metričnimi dimenzijami in ugotovili, da se ujemajo za vse $n \in [9, \ldots, 80]$.
    
    \item Za $m = 6$: Podobno kot pri $m = 5$, sva na podlagi izrzačunanih metričnih dimenzij, poskušali najti funkcijo,
    ki bi jih opisala. Tu nisva imeli take sreče, saj nama ni uspelo najti funkcije, ki bi natančno opisala vse izračunane dimenzije.
    Poskušali sva s funkcijo $dim(C_n(1, \ldots , n-6)) = \lfloor \frac{9n}{25} + b \rfloor$ za različne vrednosti $b$, vendar nobena ni ustrezala vsem izračunanim dimenzijam.
    Za $b = 1$ so bile razlike najmanjše, zato sva tej funkciji posvetili največ pozornosti. Definirali sva funkcijo:
    \[
        \dim(C_n(1,2,\dots,n - 6)) =
        \begin{cases}
            \operatorname{int}(\operatorname{round}(\frac{9n}{25} + 1)) - 1, & \text{če } 11 \mid n, \\[8pt]
            \operatorname{int}(\operatorname{round}(\frac{9n}{25} + 1)) + 1, & \text{če } n = \{43, 54, 65, 68, 76, 79\} \\[8pt]
            \operatorname{int}(\operatorname{round}(\frac{9n}{25} + 1)),     & \text{sicer}.
        \end{cases}
    \]
    Ta približek je veljal za vse $n \in [11, \ldots, 80]$. Njegovo veljavnost sva preverili podobno kot prej.

    \item Za $m = 7$: Kot pri $ m=6$ sva poskusili s funkcijo $dim(C_n(1, \ldots , n-6)) = \lfloor \frac{9n}{25} + b \rfloor$ za $b \in \{0, 1, 2, 3, 4, 5\}$.
    Nobena ni ustrezala vsem izračunanim dimenzijam. Najboljše rezultate je dala funkcija za $b = 3$, a je odstopala pri več vrednostih $n$. Zato je nisva nadalje popravljali.
    \item za $m = 8, 9, 10$: Pri teh vrednostih nisva iskali nobene funkcije, saj sklepava na podlagi prejšnjih dveh, da je iskanje take funkcije za večje $m$ zelo zahtevno in verjetno ne bo dalo dobrih rezultatov.
\end{itemize}



\end{document}