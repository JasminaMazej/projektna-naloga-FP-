\documentclass[a4paper,12pt]{article}
\usepackage[slovene]{babel}
\usepackage[utf8]{inputenc}
\usepackage[T1]{fontenc}
\usepackage{lmodern}
\usepackage{amsmath,amsfonts}
\usepackage{enumitem}
\usepackage{mathabx}
\usepackage{fancyhdr}
\usepackage{url}
\usepackage{graphicx}
\usepackage{xcolor}
\usepackage{amsthm}
\usepackage{amssymb}

\theoremstyle{definition}
\newtheorem{definicija}{Definicija}

\title{Metrične dimenzije usmerjenih grafov}
\author{Jasmina Mazej in Lana Stojčić}

\begin{document}
\maketitle

\section*{Opis problema}

\begin{itemize}
    \item Opazovati obnašanje metrične dimenzije usmerjenih cirkulantnih grafov $C(n, d)$ za različne vrednosti $n$ in $k$.
    \item Za mnoge vrednosti $n$ in $k$ eksprimentalno določiti metrično dimenzijo grafov $C(n, d)$.
    \item Za mnoge vrednosti $n$ in $k$ izračunati metrično dimenzijo in poskusiti dokazati dane trditve.
\end{itemize}

\section*{Definicije}

Naj bo $G = (V, E)$ usmerjen graf, kjer je $V$ množica vseh vozlišč in $E$ množica vseh usmerjenih povezav v grafu.
Privzeli bomo, da je v vseh definicijah graf usmerjen.

\begin{definicija}
    Če obstaja direktna povezava iz vozlišča $u$ v $v$, potem je to najkrajša povezava med vozliščima.
    Označimo jo z $d(u, v)$. Če taka povezava obstaja, pravimo da je $v$ dosegljiv iz $u$.
    Če taka pot ne obstaja, označimo $d(u, v) = \infty$ in $v$ ni dosegljiv iz $u$.
\end{definicija}

\begin{definicija}
    Naj bodo $s, x, y \in V$. Pravimo, da vozlišče $s$ razreši par vozlišč $x, y\in V$, če sta $x$ in $y$ obe dosegljiv
    iz $s$, vendar pri različnih razdaljah. Torej velja:
    \[d(s, x) \neq d(s, y).\]
    Množica vozlišč $S$ razreši usmerjen graf $G$, če za vsak par $x, y\in V$ obstaja vsaj eno vozlišče $s\in S$,
    ki razreši $x$ in $y$. Najmanjša možna velikost množice $S$ se imenuje metrična dimenzija grafa G.
\end{definicija}

\begin{definicija}
    Usmerjene cirkulantne grafe označimo kot $C(n, d)$, kjer je $n$ število vozlišč v grafu, $d$ pa seznam generatorjev.
    Označimo jih lahko tudi kot $C_n(1,\dots , d)$. Njihova vozlišča so postavljena v krog, povezave pa si sledijo v periodičnem
    vzorcu. Vozlišča v tem grafu bomo označevali kot ${v_0, v_1, \dots, v_{n-1}}$. Vsako vozlišče je povezano z določenim drugim po fiksnem
    vzorcu, ki ga določijo generatorji. Torej množico povezav zapišemo kot: 
    \[E = \{(v_i, v_{i + k \text{  mod n}}); \text{  }1 \leq k \leq d\}\text{   za } i = \{0, 1, \dots, n-1\}.\]
\end{definicija}

\section*{Celoštevilski linearni program}

\begin{verbatim}
def metricna_dimenzija_usmerjenega_grafa(graf):
    
    if not isinstance(graf, DiGraph):
        return "Napaka: Podani graf ni usmerjen graf."
    
    # vozlišča grafa
    V = graf.vertices()
    
    # Izračun razdalj med vsemi pari vozlišč
    razdalje = {u: {v: graf.distance(u, v) for v in V} for u in V}
    
    # linearni program
    lp = MixedIntegerLinearProgram(maximization=False)
    x = lp.new_variable(binary=True)  # Ustvarjanje binarnih spremenljivk za vsako vozlišče
    
    # Cilj: minimizirati vsoto vseh x[v]
    lp.set_objective(sum(x[v] for v in V))
    
    # Preverjanje, ali graf izpolnjuje pogoje
    for u in V:
        for v in V:
            if u != v:
                vozlisca = [
                    w for w in V
                    if razdalje[w][u] != razdalje[w][v] and
                    razdalje[w][u] < infinity and
                    razdalje[w][v] < infinity
                ]
                
                # Če za par u, v ne obstaja ustrezno w, ne moremo izračunati metrične dimenzije
                if not vozlisca:
                    return f"Metrične dimenzije ni mogoče določiti: za par vozlišč ({u}, {v}) ne obstaja ustrezno vozlišče."
                
                # Dodamo omejitev, če obstajajo ustrezna vozlišča w
                lp.add_constraint(sum(x[w] for w in vozlisca) >= 1)
    
    # Rešitev linearnega programa
    lp.solve()
    
    # Pridobimo rezultate
    razresljiva_mnozica = [v for v in V if lp.get_values(x[v]) == 1]
    return razresljiva_mnozica, len(razresljiva_mnozica)


def clockwise_circulant_graph(n, d):
    odmiki = list(range(1, d + 1)) 
    G = digraphs.Circulant(n, odmiki)
    #plot = G.plot(layout="circular", vertex_size=300, vertex_color="skyblue", edge_color="black", 
                  #vertex_labels=True)
    #plot.show() #zanima naju samo dimenzija, zato sva odstranili del, ki nariše graf
    return G
\end{verbatim}

\section*{Opis programa}

\begin{enumerate}
    \item Preverjanje, ali je graf usmerjen.
    \item Izračun vseh razdalj med vozlišči.
    \item Postavitev celoštevilske optimizacije.
    \item Dodajanje omejitev za vsak par vozlišč.
    \item Reševanje ILP in izpis rezultata.
\end{enumerate}


\section*{Uvod in opis problema}

Metrična dimenzija grafa je pomembna invariantna količina v teoriji grafov, ki meri najmanjšo množico vozlišč, s pomočjo katerih lahko z razdaljami enolično ločimo vsa vozlišča grafa.

V projektu obravnavamo \textbf{usmerjene cirkulantne grafe}
\[
C_n(1,2,\dots,k),
\]
ter eksperimentalno preverjamo znane teoretične ocene in domneve o njihovi metrični dimenziji.

Znano je, da velja:
\[
\dim(C_n(1,2,\dots,k)) = k \quad \text{za } n > 2(k-1)^2.
\]

Naš cilj je preveriti:
\begin{itemize}
\item ali lahko obstaja manjša meja $f(k) < 2(k-1)^2$,
\item preveriti domnevo za primer $k=n-4$,
\item ter eksperimentalno raziskati primer $k \ge 5$.
\end{itemize}

%-------------------------------------------------
\section*{Definicije}

Naj bo $G = (V,E)$ usmerjen graf.

\textbf{Definicija.}
Če obstaja usmerjena pot iz $u$ v $v$, razdaljo $d(u,v)$ definiramo kot dolžino najkrajše take poti. Če poti ni, je $d(u,v)=\infty$.

\textbf{Definicija.}
Vozlišče $s$ razreši par vozlišč $x,y$, če velja
\[
d(s,x) \neq d(s,y).
\]

\textbf{Definicija.}
Množica $S$ razreši graf $G$, če za vsak par $x,y$ obstaja $s\in S$, ki razreši ta par. Najmanjša taka množica ima moč $\dim(G)$ in jo imenujemo \emph{metrična dimenzija} grafa.

\textbf{Definicija.}
Usmerjeni cirkulantni graf $C_n(1,2,\dots,k)$ ima vozlišča
$\{0,1,\dots,n-1\}$ in povezave oblike
\[
i \to i+r \pmod n, \quad r\in\{1,2,\dots,k\}.
\]


%-------------------------------------------------
\section*{Problem 1: Iskanje funkcije $f(k)$}

Naloga:
\[
\text{za } k\ge 3 \text{ poišči } f(k)<2(k-1)^2 \text{, da velja } \dim(C_n(1,\dots,k))=k.
\]

Eksperimentalno smo s pomočjo izračunov preverili več vrednosti $k$ in $n$.

\textbf{Rezultati (iz Excel tabele):}
\begin{itemize}
\item za $k=3$: $f(3) = \underline{\hspace{1cm}}$,
\item za $k=4$: $f(4) = \underline{\hspace{1cm}}$,
\item za $k=5$: $f(5) = \underline{\hspace{1cm}}$.
\end{itemize}

(V te vrstice samo prepišeš konkretne vrednosti iz Excel tabele.)

Ugotovimo, da v vseh obravnavanih primerih velja:
\[
f(k) < 2(k-1)^2.
\]

%-------------------------------------------------
\section*{5. Znane spodnje meje}

Velja:
\[
\dim(C_n(1,\dots,n-1)) \ge n-1,\quad n\ge 3,
\]
\[
\dim(C_n(1,\dots,n-2)) \ge \left\lceil\frac{n}{2}\right\rceil,\quad n\ge 4,
\]
\[
\dim(C_n(1,\dots,n-3)) \ge \left\lceil\frac{n}{2}\right\rceil,\quad n\ge 5.
\]

Tudi te ocene se skladajo z našimi numeričnimi rezultati.

%-------------------------------------------------
\section*{6. Conjecture 2: Primer $k = n-4$}

Domneva:
\[
\dim(C_n(1,2,\dots,n-4)) = \left\lceil\frac{2n}{5}\right\rceil, \quad n\ge 7.
\]

S pomočjo izračunov za $7 \le n \le 40$ smo ugotovili:

\begin{center}
\emph{Domneva velja za vse preizkušene vrednosti $n$.}
\end{center}

(Tu lahko po želji vstaviš tudi posnetek zaslona Excel tabele.)

%-------------------------------------------------
\section*{7. Splošni primer $k \ge 5$}

Za vrednosti $k=5,6,7,8,9,10$ smo eksperimentalno izračunali vrednosti
\[
\dim(C_n(1,\dots,n-k)).
\]

Iz rezultatov (glej Excel datoteko) opazimo:
\begin{itemize}
\item metrična dimenzija narašča približno linearno z $n$,
\item večje kot je $k$, počasnejša je rast,
\item zdi se, da velja približna zveza:
\[
\dim(C_n(1,\dots,n-k)) \approx \alpha_k n.
\]
\end{itemize}

%-------------------------------------------------
\section*{8. Zaključek}

V nalogi smo uspešno:
\begin{itemize}
\item implementirali izračun metrične dimenzije,
\item eksperimentalno določili boljše meje za Problem 1,
\item potrdili veljavnost domneve za $k=n-4$,
\item ter analizirali obnašanje za splošne vrednosti $k\ge 5$.
\end{itemize}

Projekt potrjuje, da je kombinacija ILP in eksperimentalne analize učinkovito orodje pri preučevanju metrične dimenzije.


\end{document}