\documentclass[a4paper,12pt]{article}
\usepackage[slovene]{babel}
\usepackage[utf8]{inputenc}
\usepackage[T1]{fontenc}
\usepackage{lmodern}
\usepackage{amsmath,amsfonts}
\usepackage{enumitem}
\usepackage{mathabx}
\usepackage{fancyhdr}
\usepackage{url}
\usepackage{graphicx}
\usepackage{xcolor}
\usepackage{amsthm}
\usepackage{amssymb}

\theoremstyle{definition}
\newtheorem{definicija}{Definicija}

\title{Metrične dimenzije usmerjenih grafov}
\author{Jasmina Mazej in Lana Stojčić}

\begin{document}
\maketitle

\section*{Opis problema}

\begin{itemize}
    \item Napisati celoštevilski linearni program za izračun metrične dimenzije usmerjenih grafov.
    \item Opazovati obnašanje metrične dimenzije usmerjenih cirkulantnih grafov $C(n, d)$ za različne vrednosti $n$ in $k$.
    \item Za mnoge majhne vrednosti $n$ in $k$ eksprimentalno določiti metrično dimenzijo grafov $C(n, d)$.
    \item Za mnoge vrednosti $n$ in $k$ izračunati metrično dimenzijo in poskusiti dokazati dane trditve.
\end{itemize}

\section*{Definicije}

Naj bo $G = (V, E)$ usmerjen graf, kjer je $V$ množica vseh vozlišč in $E$ množica vseh usmerjenih povezav v grafu.
Privzeli bomo, da je v vseh definicijah graf usmerjen.

\begin{definicija}
    Če obstaja direktna povezava iz vozlišča $u$ v $v$, potem je to najkrajša povezava med vozliščima.
    Označimo jo z $d(u, v)$. Če taka povezava obstaja, pravimo da je $v$ dosegljiv iz $u$.
    Če taka pot ne obstaja, označimo $d(u, v) = \infty$ in $v$ ni dosegljiv iz $u$.
\end{definicija}

\begin{definicija}
    Naj bodo $s, x, y \in V$. Pravimo, da vozlišče $s$ razreši par vozlišč $x, y\in V$, če sta $x$ in $y$ obe dosegljiv
    iz $s$, vendar pri različnih razdaljah. Torej velja:
    \[d(s, x) \neq d(s, y).\]
    Množica vozlišč $S$ razreši usmerjen graf $G$, če za vsak par $x, y\in V$ obstaja vsaj eno vozlišče $s\in S$,
    ki razreši $x$ in $y$. Najmanjša možna velikost množice $S$ se imenuje metrična dimenzija grafa G.
\end{definicija}

\begin{definicija}
    Usmerjene cirkulantne grafe označimo kot $C(n, d)$, kjer je $n$ število vozlišč v grafu, $d$ pa seznam generatorjev.
    Označimo jih lahko tudi kot $C_n(1,\dots , d)$. Njihova vozlišča so postavljena v krog, povezave pa si sledijo v periodičnem
    vzorcu. Vozlišča v tem grafu bomo označevali kot ${v_0, v_1, \dots, v_{n-1}}$. Vsako vozlišče je povezano z določenim drugim po fiksnem
    vzorcu, ki ga določijo generatorji. Torej množico povezav zapišemo kot: 
    \[E = \{(v_i, v_{i + k \text{  mod n}}); \text{  }1 \leq k \leq d\}\text{   za } i = \{0, 1, \dots, n-1\}.\]
\end{definicija}

\section*{Celoštevilski linearni program}

\begin{verbatim}
def metricna_dimenzija_usmerjenega_grafa(graf):
    
    if not isinstance(graf, DiGraph):
        return "Napaka: Podani graf ni usmerjen graf."
    
    # vozlišča grafa
    V = graf.vertices()
    
    # Izračun razdalj med vsemi pari vozlišč
    razdalje = {u: {v: graf.distance(u, v) for v in V} for u in V}
    
    # linearni program
    lp = MixedIntegerLinearProgram(maximization=False)
    x = lp.new_variable(binary=True)  # Ustvarjanje binarnih spremenljivk za vsako vozlišče
    
    # Cilj: minimizirati vsoto vseh x[v]
    lp.set_objective(sum(x[v] for v in V))
    
    # Preverjanje, ali graf izpolnjuje pogoje
    for u in V:
        for v in V:
            if u != v:
                vozlisca = [
                    w for w in V
                    if razdalje[w][u] != razdalje[w][v] and
                    razdalje[w][u] < infinity and
                    razdalje[w][v] < infinity
                ]
                
                # Če za par u, v ne obstaja ustrezno w, ne moremo izračunati metrične dimenzije
                if not vozlisca:
                    return f"Metrične dimenzije ni mogoče določiti: za par vozlišč ({u}, {v}) ne obstaja ustrezno vozlišče."
                
                # Dodamo omejitev, če obstajajo ustrezna vozlišča w
                lp.add_constraint(sum(x[w] for w in vozlisca) >= 1)
    
    # Rešitev linearnega programa
    lp.solve()
    
    # Pridobimo rezultate
    razresljiva_mnozica = [v for v in V if lp.get_values(x[v]) == 1]
    return razresljiva_mnozica, len(razresljiva_mnozica)


def clockwise_circulant_graph(n, d):
    odmiki = list(range(1, d + 1)) 
    G = digraphs.Circulant(n, odmiki)
    #plot = G.plot(layout="circular", vertex_size=300, vertex_color="skyblue", edge_color="black", 
                  #vertex_labels=True)
    #plot.show() #zanima naju samo dimenzija, zato sva odstranili del, ki nariše graf
    return G
\end{verbatim}

\section*{Opis programa}

\begin{enumerate}
    \item Preverjanje, ali je graf usmerjen.
    \item Izračun vseh razdalj med vozlišči.
    \item Postavitev celoštevilske optimizacije.
    \item Dodajanje omejitev za vsak par vozlišč.
    \item Reševanje ILP in izpis rezultata.
\end{enumerate}


\end{document}