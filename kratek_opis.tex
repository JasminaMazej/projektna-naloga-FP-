\documentclass[a4paper,12pt]{article}
\usepackage[slovene]{babel}
\usepackage[utf8]{inputenc}
\usepackage[T1]{fontenc}
\usepackage{lmodern}
\usepackage{amsmath,amsfonts}
\usepackage{enumitem}
\usepackage{mathabx}
\usepackage{fancyhdr}
\usepackage{url}
\usepackage{graphicx}
\usepackage{xcolor}
\usepackage{amsthm}
\usepackage{amssymb}

\theoremstyle{definition}
\newtheorem{definicija}{Definicija}

\title{Metrične dimenzije usmerjenih grafov}
\author{Jasmina Mazej in Lana Stojčić}

\begin{document}
\maketitle

\section{Uvod}
V tej nalogi, si bova pomagali s spodaj navedenimi definicijami. Na podlagi teh definicij, sva oblikovali
razmislek, kako se bova lotili reševanja nalog.

\section{Definicije in osnovni pojmi}

Naj bo $G = (V, E)$ usmerjen graf, kjer je $V$ množica vseh vozlišč in $E$ množica vseh usmerjenih povezav v grafu.
Privzeli bomo, da je v vseh definicijah graf usmerjen.

\begin{definicija}
    Če obstaja direktna povezava iz vozlišča $u$ v $v$, potem je to najkrajša povezava med vozliščima.
    Označimo jo z $d(u, v)$. Če taka povezava obstaja, pravimo da je $v$ dosegljiv iz $u$.
    Če taka pot ne obstaja, označimo $d(u, v) = \infty$ in $v$ ni dosegljiv iz $u$.
\end{definicija}

\begin{definicija}
    Naj bodo $s, x, y \in V$. Pravimo, da vozlišče $s$ razreši par vozlišč $x, y\in V$, če sta $x$ in $y$ obe dosegljiv
    iz $s$, vendar pri različnih razdaljah. Torej velja:
    \[d(s, x) \neq d(s, y).\]
    Množica vozlišč $S$ razreši usmerjen graf $G$, če za vsak par $x, y\in V$ obstaja vsaj eno vozlišče $s\in S$,
    ki razreši $x$ in $y$. Najmanjša možna velikost množice $S$ se imenuje metrična dimenzija grafa G.
\end{definicija}

\begin{definicija}
    Usmerjene cirkulantne grafe označimo kot $C(n, d)$, kjer je $n$ število vozlišč v grafu, $d$ pa seznam generatorjev.
    Označimo jih lahko tudi kot $C_n(1,\dots , d)$. Njihova vozlišča so postavljena v krog, povezave pa si sledijo v periodičnem
    vzorcu. Vozlišča v tem grafu bomo označevali kot ${v_0, v_1, \dots, v_{n-1}}$. Vsako vozlišče je povezano z določenim drugim po fiksnem
    vzorcu, ki ga določijo generatorji. Torej množico povezav zapišemo kot: 
    \[E = \{(v_i, v_{i + k \text{  mod n}}); \text{  }1 \leq k \leq d\}\text{   za } i = \{0, 1, \dots, n-1\}.\]
\end{definicija}

\section{Postopek reševanja}
\subsection{Naloga 1}
Vemo, da je $dim(C_n(1, \dots, k)) = k$ za $n > 2(k-1)^2$. Zanima nas, ali za $k \geq 3$ obstaja $f(k) < 2(k-1)^2$,
da še vedno velja $dim(C_n(1, \dots, k)) = k$ pri pogoju $n > f(k)$.\\

Problema se bova lotili eksprimentalno na sledeč način:
\begin{itemize}
    \item Vzeli bova funkcijo za izračun metrične dimenzije cirkulantnih grafov od Maše Popovič in preverili njeno delovanje. 
    V primeru pravilnega delovanja, jo bova nadalje uporabljali v svoji nalogi.
    \item Izbrali bova čim več različnih vrednost za $n$ in  $k \geq 3$.
    \item Izbrali bova različne funkcije za $f(k)$.
\end{itemize}
Osredotočili se bova na poljubno izbrane vrednosti za $n \geq 4$ in $k \geq 3$. Ne splača se jemati $n < 4$, saj
imamo manj vozlišč kot povezav.
Za izbiro $f(k)$, pa si bova izbrali poljubno manjšo kvadratno funkcijo, in opazovali, kaj se dogaja z vrednostmi. Ogledali si bova tudi, kaj se 
zgodi, če za $f(k)$ vzamemo linearno funkcijo.\\


\subsection{Naloga 2}
Vemo:
\begin{itemize}
    \item $dim(C_n(1, 2, \ldots, n-1)) \geq n-1 \quad n \geq 3$
    \item $dim(C_n(1, 2, \ldots, n-2)) = \lfloor \frac{n}{2} \rfloor \quad n \geq 4$
    \item $dim(C_n(1, 2, \ldots, n-3)) = \lfloor \frac{n}{2} \rfloor \quad n \geq 5$
\end{itemize}

Za $n \geq 7$ imamo $dim(C_n(1, \dots, n - 4)) = \lceil \frac{2n}{5} \rceil$.
To pomeni, da bo dimenzija enaka navzgor zaokroženmu celemu številu (npr. $4,4$ bo enaka 5).


Raziskali bova, ali obstajajo podobni rezultati za $dim(C_n(1,\dots, n - k))$ za $k \geq 5$.
Opazimo, da mora biti $n > 5$, saj smo predpostavili,
da število generatorjev ne more biti negativno. Torej moramo za vsak $k$ izbrati dovolj velik $n$.


Za poljubno izbrane $k$, bova izračunali metrično dimenzijo pri različnih vrednostih $n$. Opazovali bova, kaj se dogaja
z metrično dimenzijo in tako podatke shranjevali. Najin cilj je poiskati oceno za metrično dimenzijo pri fiksnih $k$
za različne vrednosti $n$.

\end{document}